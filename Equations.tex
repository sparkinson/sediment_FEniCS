\documentclass[11pt,a4paper]{article}
\usepackage{dsfont}
\usepackage{graphicx}
\usepackage{subfigure}
\usepackage{pdflscape}
\usepackage{float}
\usepackage{natbib}
\usepackage{amsmath}
\usepackage[top=2cm,bottom=2cm,left=2.2cm,right=2.2cm]{geometry}
\usepackage{setspace}
\usepackage{indentfirst}
\usepackage[font=small,format=plain,labelfont=bf,up,%% textfont=it
  ,up]{caption}
\usepackage{tikz}
\usetikzlibrary{calc,3d}
\setlength{\parindent}{10pt}
\setlength{\parskip}{7pt}
\usepackage{upgreek}
\usepackage{breqn}
\setlength{\belowdisplayskip}{6pt plus 2pt minus 1pt}
\setlength{\belowdisplayshortskip}{6pt plus 1pt minus 1pt}
\setlength{\abovedisplayskip}{6pt plus 2pt minus 1pt}
\setlength{\abovedisplayshortskip}{6pt plus 1pt minus 1pt}

%% \doublespacing    

\bibliographystyle{plainnat}

\title{}
\author{Samuel Parkinson}
\date{\today}

\begin{document}
\begin{dgroup*}
  \begin{dmath}
    \dot{u} + u \cdot \nabla u - \nabla \cdot \sigma = f - \nabla p 
  \end{dmath}
  \begin{dmath}
    \nabla u = 0 
  \end{dmath}
  \begin{dmath}
    \sigma = \nu \left( \nabla u + \left(\nabla u \right)^\top \right) 
  \end{dmath}
\end{dgroup*}

Several authors seem to neglect $\sigma$ in this form and just have $\sigma = \nabla u$ ??

%% \begin{dgroup*}
%%   \begin{dmath}
%%     \int u \cdot \nabla u = u \cdot u - \int \nabla \cdot \left(u \right) \cdot u
%%   \end{dmath}
%%   \begin{dmath}
%%     u \cdot \nabla u = \nabla \left( u \cdot u \right) - \left( \nabla \cdot u \right) \cdot u
%%   \end{dmath}
%% \end{dgroup*}

\section{Weak form:}
\begin{dmath}
  \int_\Omega v \dot{u} \; d\Omega + \int_\Omega v u \cdot \nabla u \; d\Omega - \int_\Omega v \nabla \cdot \sigma \; d\Omega = \int_\Omega v f \; d\Omega - \int_\Omega v \nabla p \; d\Omega
\end{dmath}

\begin{dgroup*}
  \begin{dmath}
     \int v \nabla \cdot \sigma = v \sigma - \int \nabla v \cdot \sigma
  \end{dmath}
  \begin{dmath}
    v \nabla \cdot \sigma = \nabla \left(v \sigma \right) - \nabla v \cdot \sigma
  \end{dmath}
\end{dgroup*}

\begin{dmath}
  \int_\Omega \nabla \left(v \sigma \right) \; d\Omega = \int_\Upgamma \left(v \sigma \right) \cdot n \; d\Upgamma = \int_\Upgamma v \left( \sigma \cdot n \right) \; d\Upgamma
\end{dmath}

\begin{dgroup*}
  \begin{dmath}
    \int_\Omega v \dot{u} \; d\Omega + \int_\Omega v u \cdot \nabla u \; d\Omega + \int_\Omega \nabla v \cdot \sigma \; d\Omega = \int_\Omega v f \; d\Omega - \int_\Omega v \nabla p \; d\Omega + \int_\Upgamma v g \; d\Upgamma
  \end{dmath}
  \begin{dmath}
    g = \sigma \cdot n
  \end{dmath}
\end{dgroup*}

%% \begin{dgroup*}
%%   \begin{dmath}
%%     \int v \nabla p = vp - \int \nabla \cdot v p
%%   \end{dmath}
%%   \begin{dmath}
%%     v \nabla p = \nabla \cdot \left( vp \right) - \left(\nabla \cdot v \right) p
%%   \end{dmath}
%% \end{dgroup*}

\begin{dmath}
  \int_\Omega q \nabla \cdot u \; d\Omega = 0
\end{dmath}

\section{Time discretisation:}

\begin{dmath}
  \dot{u} = \frac{u_n - u_{n-1}}{\Delta t}
\end{dmath}

\begin{dmath}
  \int_\Omega v \frac{u_n - u_{n-1}}{\Delta t} \; d\Omega + \int_\Omega v \tilde{u} \cdot \nabla \bar{u} \; d\Omega + \int_\Omega \nabla v \cdot \sigma_{n-\alpha} \; d\Omega = \int_\Omega v f \; d\Omega + \int_\Omega v \nabla p \; d\Omega + \int_\Upgamma v g \; d\Upgamma
\end{dmath}
\begin{dmath}
  \sigma_{n-\alpha} = \nu \left( \nabla u_{n-\alpha} + \left(\nabla u_{n-\alpha} \right)^\top \right) 
\end{dmath}
\begin{dmath}
  \int_\Omega q \nabla \cdot u_{n-\alpha} \; d\Omega = 0
\end{dmath}


\begin{dmath}
  u_{n-\alpha} = \alpha u_{n-1} + \left( 1 - \alpha \right) u_{n}
\end{dmath}
\begin{dmath}
  u_{n-\alpha_{nl}} = \alpha_{nl} u_{n-1} + \left( 1 - \alpha_{nl} \right) u_{n*}
\end{dmath}
\begin{dgroup*}
\intertext{Explicit Adams Bashforth}
\begin{dmath}
  \bar{u} = \frac{3}{2} u_{n-1} - \frac{1}{2} u_{n-2}
\end{dmath}, 
\begin{dmath}
  \tilde{u} = \frac{3}{2} u_{n-1} - \frac{1}{2} u_{n-2}
\end{dmath}. 
\intertext{Forward Euler}
\begin{dmath}
  \bar{u} = u_{n-\alpha}
\end{dmath}, 
\begin{dmath}
  \tilde{u} = u_{n-1}
\end{dmath}. 
\intertext{Implicit Adams Bashforth}
\begin{dmath}
  \bar{u} = u_{n-\alpha}
\end{dmath}, 
\begin{dmath}
  \tilde{u} = \frac{3}{2} u_{n-1} - \frac{1}{2} u_{n-2}
\end{dmath}. 
\intertext{Impicit - as Fluidity}
\begin{dmath}
  \bar{u} = u_{n-\alpha}
\end{dmath}, 
\begin{dmath}
  \tilde{u} = u_{n-\alpha_{nl}} 
\end{dmath}. 
\end{dgroup*}
%% \begin{dmath}
%%   \int_\Omega v u \; d\Omega + \int_\Omega v u \cdot \nabla u \; d\Omega + \int_\Omega \nabla v \cdot \sigma \; d\Omega = \int_\Omega v f \; d\Omega + \int_\Omega v \nabla p \; d\Omega + \int_\Upgamma v g \; d\Upgamma
%% \end{dmath}

Top three of these time discretisation schemes lead to a completely linear problem - no need for iteration if using fully coupled equation. ???

\section{Pressure/Conservation:}

\subsection{Fully coupled:}

\begin{dmath}
  \int_\Omega v \frac{u_n - u_{n-1}}{\Delta t} \; d\Omega + \int_\Omega v \tilde{u} \cdot \nabla \bar{u} \; d\Omega + \int_\Omega \nabla v \cdot \sigma_{n-\alpha} \; d\Omega - \int_\Omega v f \; d\Omega + \int_\Omega v \nabla p \; d\Omega - \int_\Upgamma v g \; d\Upgamma = 0
\end{dmath}
\begin{dmath}
  \int_\Omega v  \left(u_n - u_{n-1} \right) \; d\Omega + \Delta t \int_\Omega v \tilde{u} \cdot \nabla \bar{u} \; d\Omega + \Delta t \int_\Omega \nabla v \cdot \sigma_{n-\alpha} \; d\Omega - \Delta t \int_\Omega v f \; d\Omega + \Delta t \int_\Omega v \nabla p \; d\Omega - \Delta t \int_\Upgamma v g \; d\Upgamma = 0
\end{dmath}

\begin{dgroup*}
  \begin{dmath}
    \int v \nabla p = vp - \int \nabla \cdot v p
  \end{dmath}
  \begin{dmath}
    v \nabla p = \nabla \cdot \left( vp \right) - \left(\nabla \cdot v \right) p
  \end{dmath}
\end{dgroup*}

\begin{dmath}
  \int_\Omega \nabla \cdot \left( vp \right) \; d\Omega = \int_\Upgamma \left(vp \right) \cdot n \; d\Upgamma
\end{dmath}
\begin{dmath}
  \int_\Omega v \left(u_n - u_{n-1} \right) \; d\Omega + \Delta t \int_\Omega v \tilde{u} \cdot \nabla \bar{u} \; d\Omega + \Delta t \int_\Omega \nabla v \cdot \sigma_{n-\alpha} \; d\Omega - \Delta t \int_\Omega v f \; d\Omega-  \Delta t \int_\Omega \left(\nabla \cdot v \right) p \; d\Omega - \Delta t \int_\Upgamma v h \; d\Upgamma = 0
\end{dmath}

\begin{dmath}
  h = g + p \cdot n
\end{dmath}

\begin{dmath}
  \int_\Omega q \nabla \cdot u_{n-\alpha} \; d\Omega = 0
\end{dmath}

Therefore:
\begin{dmath}
  \int_\Omega v \left(u_n - u_{n-1} \right) \; d\Omega + \Delta t \int_\Omega v \tilde{u} \cdot \nabla \bar{u} \; d\Omega + \Delta t \int_\Omega \nabla v \cdot \sigma_{n-\alpha} \; d\Omega - \Delta t \int_\Omega v f \; d\Omega-  \Delta t \int_\Omega \left(\nabla \cdot v \right) p \; d\Omega + \int_\Omega q \nabla \cdot u_{n-\alpha} \; d\Omega - \Delta t \int_\Upgamma v h \; d\Upgamma = 0
\end{dmath}

Is this right?? What do I do about the dirichlet pressure boundary condition?

\subsection{Incremental pressure correction (IPCS):}

Tentative velocity step:
\begin{dmath}
  \int_\Omega v \frac{u_* - u_{n-1}}{\Delta t} \; d\Omega + \int_\Omega v \tilde{u} \cdot \nabla \bar{u} \; d\Omega + \int_\Omega \nabla v \cdot \sigma_{n-\alpha} \; d\Omega - \int_\Omega v f \; d\Omega + \int_\Omega v \nabla p_{n-\frac{1}{2}} \; d\Omega - \int_\Upgamma v g \; d\Upgamma = 0
\end{dmath}

Corrected velocity:
\begin{dmath}
  \int_\Omega v \frac{u_n - u_{n-1}}{\Delta t} \; d\Omega + \int_\Omega v \tilde{u} \cdot \nabla \bar{u} \; d\Omega + \int_\Omega \nabla v \cdot \sigma_{n-\alpha} \; d\Omega - \int_\Omega v f \; d\Omega + \int_\Omega v \nabla p_{n+\frac{1}{2}} \; d\Omega - \int_\Upgamma v g \; d\Upgamma = 0
\end{dmath}

(26) - (25):
\begin{dmath}
  \int_\Omega v \frac{u_n - u_*}{\Delta t} \; d\Omega + \int_\Omega v \nabla \left(p_{n+\frac{1}{2}} - p_{n-\frac{1}{2}} \right) \; d\Omega = 0
\end{dmath}

left multiply by \begin{math}\nabla q \frac{1}{v}\end{math}:

\begin{dmath}
  \int_\Omega \nabla q \cdot \frac{u_n - u_*}{\Delta t} \; d\Omega + \int_\Omega \nabla q \cdot \nabla \left(p_{n+\frac{1}{2}} - p_{n-\frac{1}{2}} \right) \; d\Omega = 0
\end{dmath}
\begin{dmath}
  \int_\Omega \nabla q \cdot \frac{u_n}{\Delta t} \; d\Omega - \int_\Omega \nabla q \cdot \frac{u_*}{\Delta t} \; d\Omega + \int_\Omega \nabla q \cdot \nabla \left(p_{n+\frac{1}{2}} - p_{n-\frac{1}{2}} \right) \; d\Omega = 0
\end{dmath}

Where do I go from here???

%% \begin{dmath}
%%   \int_\Omega q \nabla \cdot u \; d\Omega = 0
%% \end{dmath}

%% \begin{dgroup*}
%%   \begin{dmath}
%%     \int q \nabla \cdot u = q \cdot u - \int \nabla q \cdot u 
%%   \end{dmath}
%%   \begin{dmath}
%%     \nabla \left(q \cdot u \right) - \nabla q \cdot u 
%%   \end{dmath}
%% \end{dgroup*}

%% \begin{dmath}
%%   \int_\Omega  \nabla \left(q \cdot u \right) \; d\Omega = \int_\Upgamma \left( q \cdot u \right) \cdot n \; d\Upgamma = \int_\Upgamma q u \cdot n \; d\Upgamma
%% \end{dmath}

%% \begin{dmath}
%%   \int_\Omega \nabla q \cdot u \; d\Omega - \int_\Upgamma q u \cdot n \; d\Upgamma = 0
%% \end{dmath}

%% \begin{dmath}
%%   \left(\int_\Omega \nabla q \cdot u \; d\Omega - \int_\Upgamma q u \cdot n \; d\Upgamma \right)^\top = \int_\Omega \nabla p \cdot v \; d\Omega - \int_\Upgamma p v \cdot n \; d\Upgamma 
%% \end{dmath}



\bibliography{/data/sp911/documents/Bibliography/Bibliography}

\end{document}
